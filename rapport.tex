\documentclass[a4paper]{article}


\usepackage[utf8]{inputenc}
\usepackage[T1]{fontenc}
\usepackage{textcomp}
\usepackage{mathtools,amssymb,amsthm}
\usepackage{lmodern}
\usepackage{geometry}
\geometry{hmargin=1cm,vmargin=1cm}
\usepackage{amsmath,amsfonts,amssymb}
\usepackage{tikz}
\usepackage{romanbar}
\usepackage{lipsum}
\usepackage{boxedminipage}
\usepackage{eso-pic}
\usepackage{xcolor}
\usepackage{blindtext}
\usepackage{titlesec}


\begin{document}


%\AddToShipoutPicture{%
 % \AtPageLowerLeft{%
  %  \rotatebox{90}{\colorbox{gray!20}{%
   %   \begin{minipage}{\paperheight}\sffamily 
    %  \hspace*{\stretch{1}} "Projet S5".\hspace*{\stretch{1}}
     % \end{minipage}%
   % }}%
 % }%
%}

\title{ Enseirb Matmeca \\ Rapport de projet S5 \\ MANSUBA }
\author{MOHAMMED BOUHAJA ET AMIRA ELOUAZZANI}
\maketitle



\def\arete{3} \def\epaisseur{5} \def\rayon{2}
\newcommand{\ruban}{(0,0)
++(0:0.57735*\arete-0.57735*\epaisseur+2*\rayon)
++(-30:\epaisseur-1.73205*\rayon)
arc (60:0:\rayon) -- ++(90:\epaisseur)
arc (0:60:\rayon) -- ++(150:\arete)
arc (60:120:\rayon) -- ++(210:\epaisseur)
arc (120:60:\rayon) -- cycle}


\begin{tikzpicture}[very thick,top color=white,bottom color=gray]
\shadedraw \ruban;
\shadedraw [rotate=120] \ruban;
\shadedraw [rotate=-120] \ruban;
\draw (-60:4) node[scale=5,rotate=30]{S{\color{orange}\textit{5}}};
\draw (180:4) node[scale=3,rotate=-90]{PROJET};
\clip (0,-6) rectangle (6,6); % pour croiser
\shadedraw \ruban;
\draw (60:4) node [gray,scale=3,rotate=90]{RAPPORT};
\end{tikzpicture}

\newpage

\tableofcontents

\newpage

\begin{center}
\begin{tikzpicture}
[level 1/.style={level distance=5cm,
sibling distance=5cm},
level 2/.style={level distance=5cm,
sibling distance=3cm},
level 3/.style={level distance=5cm,
sibling distance=3cm}]

\node {project}
    child {node {ensemble.h}}
    child {node {victoire.h}}
    child {node {game.h}
        child {node {mouvements.h}}
        child {node {prison.h}}
    }
    child {node {board.h}
        child {node {wolrd.h}}
        child {node {geometry.h}}
        }
    
;
\end{tikzpicture}
\end{center}

\section{INTRODUCTION}

\subsection{MANSUBA}

\subsection{Problèmatique}


\section{Hypothèse et démarches de validation}


\section{}






\end{document}