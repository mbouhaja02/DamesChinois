\documentclass[a4paper]{article}


\usepackage[utf8]{inputenc}
\usepackage[T1]{fontenc}
\usepackage{textcomp}
\usepackage{mathtools,amssymb,amsthm}
\usepackage{lmodern}
\usepackage{geometry}
\geometry{hmargin=1cm,vmargin=1cm}
\usepackage{amsmath,amsfonts,amssymb}
\usepackage{tikz}
\usepackage{romanbar}
\usepackage{lipsum}
\usepackage{boxedminipage}
\usepackage{eso-pic}
\usepackage{xcolor}
\usepackage{blindtext}
\usepackage{titlesec}


\begin{document}


%\AddToShipoutPicture{%
 % \AtPageLowerLeft{%
  %  \rotatebox{90}{\colorbox{gray!20}{%
   %   \begin{minipage}{\paperheight}\sffamily 
    %  \hspace*{\stretch{1}} "Projet S5".\hspace*{\stretch{1}}
     % \end{minipage}%
   % }}%
 % }%
%}

\title{ Enseirb Matmeca \\ Rapport de projet S5 \\ MANSUBA }
\author{MOHAMMED BOUHAJA ET AMIRA ELOUAZZANI}
\maketitle



\def\arete{3} \def\epaisseur{5} \def\rayon{2}
\newcommand{\ruban}{(0,0)
++(0:0.57735*\arete-0.57735*\epaisseur+2*\rayon)
++(-30:\epaisseur-1.73205*\rayon)
arc (60:0:\rayon) -- ++(90:\epaisseur)
arc (0:60:\rayon) -- ++(150:\arete)
arc (60:120:\rayon) -- ++(210:\epaisseur)
arc (120:60:\rayon) -- cycle}



\newpage

\tableofcontents

\newpage

\begin{center}
\begin{tikzpicture}
[level 1/.style={level distance=5cm,
sibling distance=5cm},
level 2/.style={level distance=5cm,
sibling distance=3cm},
level 3/.style={level distance=5cm,
sibling distance=3cm}]

\node {project}
    child {node {ensemble.h}}
    child {node {victoire.h}}
    child {node {game.h}
        child {node {mouvements.h}}
        child {node {prison.h}}
    }
    child {node {board.h}
        child {node {wolrd.h}}
        child {node {geometry.h}}
        }
    
;
\end{tikzpicture}
\end{center}

\section{INTRODUCTION}
\begin{comment}
MANSUBA est un jeu de plateau qui a un but similaire au jeu de Shtranj mais avec des conditions de victoire légèrement différente. 
Le nombre de joueur , le nombre de pions et la taille du plateau sont bien définis dans le fichier encadrant la géometrie du sujet (geometry.h)
BOARD : 
Le plateau de jeu et la combinaison d'un monde et d'une relation .
  WORLD : 
Le monde représente les positions accessibles pour le jeu qui est dans un sens algorithmique une liste de cases . Ce monde est initialisé par la structure world_t ...
  RELATION : 
Les voisins sont les cases environant directement une case en question et qui sont dans l'une des directions enuméres dans le fichier geometry et qui dépendent de leurs
positions dans le monde (enum dir_t) ...
une relation est l'ensemble des possibilité d'avoir un mouvement directe d'une case à l'autre dans une direction permise par les régles actuelles du jeu pour une case choisie.
 INITIALISATION : 
L'état du BOARD qu'on traite dans le début de sujet sera une grille 2D avec deux joueurs et 8 directions d'orientation .
\end{comment}
\subsection{Problèmatique}
\begin{comment}
Le but de notre projet sera de jouer une partie de jeu aléatoire, puis rendre l'algorithme de plus en plus flexible et général
et l'orienter vers la victoire. 

\end{comment}
\section{Hypothèse et démarches de validation}


\section{}






\end{document}